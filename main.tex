\documentclass[a4paper]{jsarticle}

\title{$n$次魔方陣の重心は一定である}
\author{杉崎 行優}
\date{}

\begin{document}

\maketitle
\abstract{
本記事では、$n$次魔方陣の重心が各$n$において一定となることを示す。
}

\thispagestyle{empty}

\section{$n$次方陣と$n$次魔方陣}

$n$を任意の整数とする。

縦$n$個、横$n$個のマスに$1$から$n^2$の数が$1$つずつ入れられているものを$n$次方陣と定義する。
$n$次方陣の各マス中の数字を図\ref{tab:square}のように定義する。

\begin{figure}[htb]
\begin{center}
\begin{tabular}{|c|c|c|c|}
\hline
$m_{(1,1)}$ & $m_{(1,2)}$ & $\cdots$ & $m_{(1,n)}$ \\\hline
$m_{(2,1)}$ & $m_{(2,2)}$ & $\cdots$ & $m_{(2,n)}$ \\\hline
$\vdots$ & $\vdots$ & $\ddots$ & $\vdots$ \\\hline
$m_{(n,1)}$ & $m_{(n,2)}$ & $\cdots$ & $m_{(n,n)}$ \\\hline
\end{tabular}
\end{center}
\caption{$n$次方陣における各マス中の数字}
\label{tab:square}
\end{figure}

また、$n$次方陣のうち、縦の$n$列、横の$n$列、斜めの$2$列それぞれの合計が全て等しくなるものを
$n$次魔方陣と定義する。
$1$列の合計$L$は式\ref{eqn:OneLine}によって表される。

\begin{equation}
L=\frac{1}{n}\sum_{i=1}^{n^2}i
\label{eqn:OneLine}
\end{equation}

\section{$n$次魔方陣の重心}

横方向の重心を$g_x$、縦方向の重心を$g_y$とすると、
$g_x$と$g_y$の値はそれぞれ式\ref{eqn:gx}、式\ref{eqn:gy}によって表される。
よって、$n$次魔方陣の重心が各$n$において一定となることが示された。

\begin{equation} \label{eqn:gx}
\displaystyle
g_x = \frac{\displaystyle \sum_{i=1}^n \sum_{j=1}^n m_{(i,j)} \cdot i}{\displaystyle \sum_{i=1}^n \sum_{j=1}^n m_{(i,j)}}
=\frac{\displaystyle L \cdot \sum_{i=1}^n i}{n \cdot L}
=\frac{\displaystyle \frac{1}{2} n (n+1)}{n}
=\frac{1}{2} (n+1)
\end{equation}

\begin{equation} \label{eqn:gy}
\displaystyle
g_y = \frac{\displaystyle \sum_{i=1}^n \sum_{j=1}^n m_{(i,j)} \cdot j}{\displaystyle \sum_{i=1}^n \sum_{j=1}^n m_{(i,j)}}
=\frac{\displaystyle L \cdot \sum_{j=1}^n j}{n \cdot L}
=\frac{\displaystyle \frac{1}{2} n (n+1)}{n}
=\frac{1}{2} (n+1)
\end{equation}

\end{document}
